
\chapter{DETERMINING D-H PARAMETERS}

\section{Topic}
Determine the Denavit-Hartenberg parameters for the robot model.

\section{Theory}
The Denavit-Hartenberg notation is introduced as a systematic method of describing the kinematic relationship ${}^{i-1}T_i$ using only four parameters \cite{1}:

\begin{center}
     \begin{tabular}{|c|l|l|}
          \hline
          $\alpha$ & Link twist & Describe the link itself \\
          \hline
          $a$ & Link length & \\
          \hline
          $d$ & Link offset & Describe the link's connection to neighboring link \\
          \hline
          $\theta$ & Joint angle & \\
          \hline
          \multicolumn{3}{|l|}{If the joint is:} \\
          \hline
          \multicolumn{2}{|l|}{Revolute: $\theta$ joint variable} & The other three are fixed link parameters \\
          \hline
          \multicolumn{2}{|l|}{Prismatic: $d$ joint variable} & \\
          \hline
          \end{tabular}
\end{center}



\section{Application}
We got the D-H table:

\begin{center}
\begin{tabular}{|c|c|c|c|c|}
\hline
$i$ & $a_i$ & $\alpha_i$ & $d_i$ & $\theta_i$ \\
\hline
1 & 0 & 0 & 0 & $\theta_1$ \\
2 & 0 & 0 & $d_2$ & 0 \\
3 & $\ell_3$ & 0 & 0 & 0 \\
4 & $\ell_4$ & 0 & 0 & $\theta_4$ \\
\hline
\end{tabular}
\end{center}


Limitations of the D-H notation:
\begin{align*}
     &\ell_3 = 1000 \,\mathrm{mm}\\
     &\ell_4 = 300 \,\mathrm{mm}\\
     &d_2 \in [2150; 2750] \\
     &\theta_1 \in [0^\circ; 360^\circ] \\
     &\theta_4 \in [-90^\circ; 90^\circ]
\end{align*}

