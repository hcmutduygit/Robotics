\usepackage{vntex} % Tiếng Việt
\usepackage{graphicx} % Chèn hình ảnh
\usepackage{fancyhdr} % Gói hỗ trợ tạo header và footer fancy
\usepackage{changepage} % Thay đổi lề

% Chèn code
\usepackage{listings} % Thêm gói listings để chèn code
\usepackage{xcolor} % Màu cho code
\lstset{
    language=Matlab,
    basicstyle=\footnotesize\ttfamily,
    numbers=none,
    numberstyle=\tiny\color{gray},
    stepnumber=1,
    numbersep=0.01pt,
    tabsize=2,
    breaklines=true,
    breakatwhitespace=false,
    xleftmargin=0cm, % for line numbers
    framexleftmargin=0cm, % for code frame
    keywordstyle=\color{blue},
    commentstyle=\color{green},
    stringstyle=\color{orange},
    frame=single,
    rulecolor=\color{black},
    basicstyle=\ttfamily,
}

% Footnote, reference and appendix
\usepackage[style=numeric,backend=biber]{biblatex} % Sử dụng gói biblatex
\usepackage{capt-of} %  Footnote trong caption
\usepackage[perpage]{footmisc} % Đánh số lại chú thích mỗi trang
\usepackage[toc,page]{appendix}

% Thiết lập bảng
\usepackage{array} % Gói hỗ trợ các bảng phức tạp
\usepackage{tabularx}
\usepackage{longtable} % Tạo bảng qua nhiều trang
\usepackage{cellspace}
\usepackage{diagbox} % Gói hỗ trợ tạo các ô chéo trong bảng
\usepackage{multirow}
\usepackage{makecell}
\usepackage{adjustbox}

% Thiết lập công thức toán học
\usepackage{amsmath} % Gói hỗ trợ các công thức toán học
\usepackage{amsfonts} % Gói hỗ trợ các ký hiệu toán học
\usepackage{amssymb} % Gói hỗ trợ các ký hiệu toán học
\usepackage{graphicx} % Gói hỗ trợ chèn hình ảnh
\usepackage{bm} % Chữ in đậm trong công thức toán 
\usepackage{physics}

% Thiết lập khác
\usepackage{tikz}
\usepackage{color}
\usepackage{subcaption}
\usepackage{framed}
\usepackage{float} % Để chèn hình ảnh vào đúng vị trí
\usepackage{fancyvrb} % Đưa dữ liệu dạng nguyên thủy vào

% Thiết lập kích thước
\usepackage{geometry}
\geometry{
    left=3cm,
    right=2cm,
    top=2.5cm,
    bottom=2.5cm,
}
\usepackage{hyperref} %Chèn link
\hypersetup{urlcolor=black,linkcolor=black,citecolor=black,colorlinks=true} % Màu cho các đường nét
\everymath{\color{black}}
\setlength{\headheight}{20pt}
\pagestyle{fancy}
